\documentclass[a4paper,10pt]{article}
\usepackage[utf8]{luainputenc}
\usepackage[lithuanian]{babel}

%opening
\title{Dokumentacija}
\author{}
\date{2023-12-27}

\begin{document}

\maketitle

\section*{initHtmlElement()}
\textbf{Pilnas aprašas:} \textit{HtmlElement* initHtmlElement(char* elementType)}
\\\textbf{Funkcijos apibudinimas:}
\\Sukuria html elementą, kurio tipas yra nurodomas pagal char rodyklę elementType. 
Nustato visas html elemento kintamųjų reikšmes. Inicializuoja html elemento rodyklių 
masyvą į kitus gilesnius html elementus. Grąžina rodyklę į pagrindinį html elementą.

\section*{addChild()}
\textbf{Pilnas aprašas:} \textit{HtmlElement* addChild(HtmlElement* parent, HtmlElement* child)}
\\\textbf{Funkcijos apibudinimas:}
\\Funkcija nurodytam pagrindiniui html elementui priskiria šalutinį html elementą. Jeigu pagrindinio html elemento rodyklių masyvas
yra per mažas, kad būtų galima pridėti dar vieną rodyklę į šalutinį html elementą, masyvas yra padidinamas du kartus. Pridejus dar vieną
rodyklę į šalutinį elementą, pagrindinio html elemento rodklių skaitliukas padidinamas vienetu.

\section*{freeHtmlElement()}



\end{document}
