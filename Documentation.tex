\documentclass[a4paper,10pt]{article}
\usepackage[utf8]{luainputenc}
\usepackage[lithuanian]{babel}
\usepackage{tikz}

%opening
\title{Dokumentacija}
\author{}
\date{2023-12-27}

\begin{document}

\maketitle

\section{Funkcijos}
\subsection{initHtmlElement()}
\label{initHtmlElement}
\textbf{Pilnas aprašas:} \textit{HtmlElement* initHtmlElement(char* elementType)}
\\\textbf{Funkcijos apibudinimas:}
\\Sukuria html elementą, kurio tipas yra nurodomas pagal char rodyklę elementType. 
Nustato visas html elemento kintamųjų reikšmes. Inicializuoja html elemento rodyklių 
masyvą į kitus gilesnius html elementus. Grąžina rodyklę į pagrindinį html elementą.

\subsection{addChild()}
\label{addChild}
\textbf{Pilnas aprašas:} \textit{HtmlElement* addChild(HtmlElement* parent, HtmlElement* child)}
\\\textbf{Funkcijos apibudinimas:}
\\Funkcija nurodytam pagrindiniui html elementui priskiria šalutinį html elementą. Jeigu pagrindinio html elemento rodyklių masyvas
yra per mažas, kad būtų galima pridėti dar vieną rodyklę į šalutinį html elementą, masyvas yra padidinamas du kartus. Pridejus dar vieną
rodyklę į šalutinį elementą, pagrindinio html elemento rodklių skaitliukas padidinamas vienetu.

\subsection{freeHtmlElement()}
\label{freeHtmlElement}
\textbf{Pilnas aprašas:} \textit{void freeHtmlElement(HtmlElement* htmlElement)}
\\\textbf{Funkcijos apibudinimas:}
\\Funkcija atlaisvina visas rodykles į gilesnius html elementus. Funkcija atlaisvina nurodyto elemento rodyklę.

\subsection{initHtmlPage}

\textbf{Pilnas aprašas:} \textit{HtmlPage* initHtmlPage(char* fileName)}
\\\textbf{Funkcijos apibudinimas:}
\\Funkcija inicializuoja html puslapį, sukuria .html failą pagal char rodyklę fileName. Išsaugo failo adresą htmlPage elemente.
Sukuria \textit{head} ir \textit{body} html elementus, kuriuos priskiria htmlPage elementui. Grąžina htmlPage elemento rodyklę.


\subsection{addBodyElement}
\textbf{Pilnas aprašas:} \textit{HtmlElement* addBodyElement(HtmlPage* htmlPage, HtmlElement* htmlElement)}
\\\textbf{Funkcijos apibudinimas:}
\\Prie html puslapio prideda html elementą.
\\htmlElement \rightarrow{} htmlBody \rightarrow{} htmlPage
\\Jei įvyksta klaida, funkcija grąžina NULL reikšmę. Kitu atveju rodyklę į htmlElement.

\subsection*{writeHtmlElement}
\textbf{Pilnas aprašas:} \textit{void writeHtmlElement(HtmlPage* htmlPage, HtmlElement* htmlElement, unsigned short depth)}
\\\textbf{Funkcijos apibudinimas:}
\\Išsaugoja nurodytą html elementą ir visus gilesnius html elementus einančios po jo. Išsaugojimo vieta - htmlPage elemente nurodytas failo adresas.

\subsection{void freeHtmlPage(HtmlPage* htmlPage)}
\textbf{Pilnas aprašas:} \textit{void freeHtmlPage(HtmlPage* htmlPage)}
\\\textbf{Funkcijos apibudinimas:}
\\Atlaisvina visus html elementus ir html puslapio rodyklę.

\subsection{createHtmlPage}
\textbf{Pilnas aprašas:} \textit{void freeHtmlPage(HtmlPage* htmlPage)}
\\\textbf{Funkcijos apibudinimas:}
\begin{itemize}
\item Prideda \textit{"<!DOCTYPE html>}\verb|\|\textit{n"} html failo pradžioje. 
\item Sukuria html head.
\item Sukuria html body (surašo visus html elementus).
\item Atlaisvina html puslapio elementą.
\end{itemize}



\section{Programos vizualizacija}
\tikzstyle{RectObject}=[rectangle,fill=white,draw,line width=0.5mm]
\tikzstyle{line}=[draw]
\tikzstyle{arrow}=[draw, -latex] 
\begin{tikzpicture}
\draw(-5, 1) node[RectObject] (fun1) {initHtmlElement()};
\draw(-5, 0) node[RectObject] (fun2) {addChild()};
\draw(-5, -1) node[RectObject] (fun3) {freeHtmlElement()};
\draw (0, 0) node[RectObject] (htel) {html elementai};
\draw[arrow] (fun1.east)--(htel.west);
\draw[arrow] (fun2.east)--(htel.west);
\draw[arrow] (fun3.east)--(htel.west);
\end{tikzpicture}


\end{document}
